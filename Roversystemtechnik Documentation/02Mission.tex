%###########################################################################
%
%   Einleitung
%
%###########################################################################
%
\chapter{The Mission}
\label{chap:mission}

During the observation  of Jupiter, the Galileo spacecraft did also some flybys of the Jupiter moons, \cite{Mission_01}.
The scientist gathered data from Europa, which supported the evidence of a thick icy surface.
The possibility of liquid water underneath lead astrobiologists to the assumption that extraterrestrial life could exist on Europa, \cite{Mission_02}.
That is why Europo is - beside Mars - an interesting object of research.\\

Therefore, the ESA will launch the \textit{JUICE} orbiter in 2022 to investigate Europa in more detail, \cite{Mission_03}. 
But also the NASA is developing  \textit{Europa Clipper} to get detailed information.
Additionally, they plan a lander for Europer to bring scientific instruments onto the surface. \cite{Mission_04} \cite{Mission_05}.
%There is a side mission planed DLR will perform the side mission \textsc{Technologies for Rapid Ice Penetration and Subglacial Lake Exploration}, with project coordinator Dr. Waldmann, which will take samples of the water by melting through the ice with an special testing probe.\\

Under the leadership of  Prof. Dr.-Ing. Klinkner, the Institute of Aero Space Systems started within a seminar a feasibility study about a rover system to explore Europa surface, which shall  be part of the \textit{TRIPLE} mission.
This challenge was given to five student teams in order to develop concepts, construct preliminary designs, perform analysis and make evaluations to  meet the mission objectives and fit the mandatory requirements cite. \\


This report contain the results of the Phase A study of the rover system \textsc{IN-situ Sampling and Primal Investigation Rover on Europa}  (INSPIRE).