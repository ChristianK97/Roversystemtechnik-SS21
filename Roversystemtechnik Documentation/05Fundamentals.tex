\chapter{Fundamentals}
\label{chap:Fundamentals}
For the detailed test-bed design, the corresponding physical fundamentals are required, which have to be introduced and explained accordingly. Thereby, the scientific principles involved must be outlined, the corresponding interrelationships explained and relevant parameters introduced.


\section{Mechanics}
As a first approach the mechanical fundamentals regarding this test-bed are discussed.
\label{sec:mechanics}
\subsection{Angular Momentum and Torque}
Due to the fact that actuators and RWs are torque generating and angular momentum storing devices, a fundamental understanding for the principles of angular momentum and torque is essential. This Section focuses on explaining these principles.\\


The angular momentum $L$ of a system with a moving point mass $m$ can be described with respect to the coordinate origin $r = 0$ as a vector product using the following Equation \cite{Demtroder.2017}:

\begin{equation}
\vec{L} = ( \vec{r} \times \vec{P} ) = m \cdot ( \vec{r} \times \vec{v}) .
\label{eqn:AngularMomentum}
\end{equation}

\begin{equation}
\vec{P} = m \cdot \vec{v} .
\label{eqn:Momentum}
\end{equation}

Where $P$ describes the momentum of the point mass moving on an arbitrary path $r = r(t)$ with the velocity $v$. The angular momentum is defined to be perpendicular to $r$ and $v$. The described system can be seen in \autoref{fig:AngularMomentum1}. For a constant angular momentum, the motion proceeds in a plane perpendicular to the angular momentum vector. In the case of a planar motion the angular momentum is defined to point into the direction of the plane-normal perpendicular to the plane. The vector product $(\vec{r} \times \vec{v})$ forms a right-handed coordinate system.

\begin{figure}[H]
{\centering
\includegraphics[width=1.\textwidth]{Pictures/AngularMomentum1}
\caption{Angular momentum $L$ for a planar motion of a point mass $m$ \cite{Demtroder.2017}.}
\label{fig:AngularMomentum1}
}
\end{figure}

If a uniform circle motion is observed, the constant angular momentum $L$ points into the direction of the axis through the circle center perpendicular to the circular plane. In other words, the angular momentum $L$ points into the direction of the angular velocity vector $\omega$ as described in (\ref{eqn:AngularMomentum2}):

\begin{equation}
|L| = L = m \cdot r \cdot v \cdot \sin{(r,v)} = m \cdot r \cdot v = m \cdot r^{2} \cdot \omega .
\label{eqn:AngularMomentum2}
\end{equation}

\begin{equation*}
\sin{(r,v)} = 1 \ \text{because} \ r \perp v .
\end{equation*}

As given in Equation (\ref{eqn:AngularMomentum2}), $L = \text{const}$ for an uniform angular motion as they are characterized through a constant radius $r= \text{const}$ as the arbitrary path and a constant velocity $v = \text{const}$.

The differentiation of the Equation (\ref{eqn:AngularMomentum}) with respect to the time $t$ is:


\begin{equation}
\frac{dL}{dt} = \left[ \dfrac{dr}{dt} \times p \right] + [r \times \frac{dp}{dt}] = (v \times p) + (r \times \dot{p}) = (r \times \dot{p}) .
\label{eqn:DiffAngularMomentum}
\end{equation}

\begin{equation*}
\text{because} \  v \parallel p , \ \frac{dL}{dt} = (x \times F), \ F = \frac{dp}{dt}. 
\end{equation*}

This leads to the definition of the torque $T$ resulting through the force $F$ around the origin $r =0$ acting on the mass $m$ at the position $r$. Another definition of the torque $T$ is provided through the derivation of the angular momentum $L$ with respect to time $t$.\\
Which means that $T = 0 \to L = const. \to \frac{\text{dL}}{\text{dt}} = 0$. This is known as the \textit{Law of Angular Momentum Conservation}, which applies to closed systems where no torque is generated.

\begin{equation}
T = \frac{dL}{dt} = (r \times F) .
\label{eqn:torque}
\end{equation} 

\begin{figure}[H]
{\centering
\includegraphics[width=1.\textwidth]{Pictures/AngularMomentum2}
\caption{Angular momentum $L$ for a uniform circular motion of a point mass $m$ \cite{Demtroder.2017}.}
\label{fig:AngularMomentum2}
}
\end{figure}



\subsection{Moment of Inertia (MoI)}
\label{sec:Inertia}
The MoI of a rigid body with the mass $m$ can be described by summarizing the products calculated by multiplying the mass of each element of a body by its squared distance from the axis of rotation \cite{.2003}. Hence making this relevant for any body rotating about an axis.\\
When mentioning MoI, the main inertia, which is the largest and refers to the axis of rotation of the rigid body, is the most important \cite{Messerschmid.2017}\cite{Demtroder.2017}.\\
The mass of a rigid body can be calculated through its density $\rho$ and its volume $V$:

\begin{equation}
m = \rho \cdot V .
\label{eqn:mass}
\end{equation}

While the volume $V$ of a cylinder with an outer radius $R$ and a height $h_\text{1}$ can be computed as seen in Equation (\ref{eqn:VolFull}) and of an hollow cylinder with an inner radius $r$ as seen in Equation (\ref{eqn:VolHollow}):

\begin{equation}
V_\text{Full} = \pi \cdot h_\text{1} \cdot R^{2} .
\label{eqn:VolFull}
\end{equation}

\begin{equation}
V_\text{Hollow} = \pi \cdot h_\text{1} \cdot (R^{2}-r^{2}) .
\label{eqn:VolHollow}
\end{equation}

Under the assumption of an homogeneous mass distribution inside the body the following MoI can be computed \cite{Spiegel.2013}:

\begin{itemize}
  \setlength{\itemsep}{2pt}
\item A circular full cylinder of outer radius $R$ and height $h$:
\end{itemize}

\begin{equation}
\begin{split}
I_\text{Full} = \dfrac{1}{2} \rho V_\text{Full} R^{2} \\\ =  \dfrac{1}{2} m_\text{Full} R^{2} .
\end{split}
\label{eqn:MoI1}
\end{equation}

\begin{itemize}
  \setlength{\itemsep}{2pt}
\item A circular hollow cylinder of outer radius $R$, inner radius $r$ and height $h$:
\end{itemize}

\begin{equation}
I_\text{Hollow} = \dfrac{1}{2} \rho V_\text{Hollow} (R^{2} +r^{2}) = \dfrac{1}{2} m_\text{Hollow} (R^{2}+r^{2}) .
\label{eqn:MoI2}
\end{equation}

\begin{itemize}
  \setlength{\itemsep}{2pt}
\item A circular partial cylinder consisting of an inner full cylinder of radius $r$ and height $h_\text{1}$ and an outer hollow cylinder of outer radius $R$, inner radius $r$ and height $h_\text{2}$:
\end{itemize}

\begin{equation}
I_\text{Partial} = \dfrac{1}{2} \rho V_\text{Full} r^{2} + \dfrac{1}{2} \rho V_\text{Hollow} (R^{2} +r^{2}) = \dfrac{1}{2} m_\text{Full} r^{2} + m_\text{Hollow} (R^{2}+r^{2}) .
\label{eqn:MoI3}
\end{equation}

\raggedbottom

\begin{figure}[H]
{\centering
\includegraphics[width=0.5\textwidth]{Pictures/MoI}
\caption{Cross sectional view of a full cylinder, hollow cylinder and partial cylinder. The cylinder heights $h_\text{1}$ and $h_\text{2}$ as well as the radius $R$ and $r$ are shown.}
\label{fig:MoI}
}
\end{figure}



The cross sectional view of the three corresponding cylindrical bodies are visualized in \autoref{fig:MoI}.\\
When examining these Equations, it can be seen that they depend on the mass and squarely on the radius or the mass distribution along the radius in relation to the axis of rotation. From this, an important principle for dimensioning bodies with a high MoI can be derived: As much mass as possible should be positioned at the largest possible distance from the axis of rotation.

\subsection{Rigid Body Torque and Angular Momentum}
\label{sec:rotation}
After the introduction of the MoI $I$ it is now possible to apply the physical conditions behind the rotational motion to rigid bodies. Hereby this is regarded in a 2D (\textit{Two Dimensional}) Cartesian Coordinate System \cite{Karaoglu.2020}\cite{Demtroder.2017}.\\

In a rotational motion the angle $\varphi$ is crucial for defining the angular position in in dependency to the coordinate system. The angular momentum $L$ can be defined as the angular velocity $\omega$ in relation to the moment of inertia $I$:

\begin{equation}
L = I \cdot \omega .
\label{eqn:omega}
\end{equation}

Therefore the torque $T$ can be defined as the product of the MoI $I$ and the angular or rotational acceleration $\dot{\omega}$ of the rigid body:

\begin{equation}
T = I \cdot \alpha =I \cdot \dot{\omega} .
\label{eqn:omegadot}
\end{equation}

Using this correlation the kinematic Equations of the rotational motion of a rigid body about a fixed axis with a constant acceleration $\dot{\omega}$ can be defined as:

\begin{equation}
\varphi = \dfrac{1}{2} \dot{\omega} t^{2} + \omega_{0} t + \varphi_{0}.
\label{eqn:rot1}
\end{equation}

\begin{equation}
\omega = \dot{\omega} t + \omega_{0}.
\label{eqn:rot2}
\end{equation}

\begin{equation}
\dot{\omega} = \alpha = \text{const}.
\label{eqn:rot3}
\end{equation}

\raggedbottom

The uniformly accelerated rotation is described in dependency of time $t$. If the acceleration is initiated from stillstand the initial angle as well as the initial angular velocity and acceleration are $\varphi_{0} = \omega_{0} = 0$ at the time $t = 0$.\\
In \autoref{fig:rigidbody} a platform is shown, which can be considered as a rigid body. Its MoI $I$, angular position $\varphi$, velocity $\omega$ and acceleration $\dot{\omega}$ is visualized as well as the angular momentum vector $\vec{L}$.\\




\begin{figure}[H]
{\centering
\includegraphics[width=0.6\textwidth]{Pictures/rigidbody}
\caption{Schematic illustration of a platform and its MoI $I$, angular position $\varphi$, velocity $\omega$ and acceleration $\dot{\omega}$ as well as the angular momentum vector $\vec{L}$.}
\label{fig:rigidbody}
}
\end{figure}




\subsection{Actuated Platform}
\label{sec:actuatedplatform}
If a platform and a torque generating actuator mounted onto it are considered, they can be assumed as two bodies in a closed system. The environment still affects the system in the form of friction, but the interaction between platform and actuator can be considered closed and without any losses. This interaction is governed by the $Actio=Reactio$\footnote{According to Newton's First Law \cite{Demtroder.2017}} principle, so the torque values of the platform $T_\text{Pl}$ and the RW $T_\text{RW}$ have the same value but have an opposing direction  \cite{Karaoglu.2020}:


\begin{equation}
|T_\text{Pl}| = |T_\text{RW}| .
\label{eqn:torque1}
\end{equation}

\begin{equation}
T_\text{Pl} = - T_\text{RW} .
\label{eqn:torque2}
\end{equation}

If the MoI and the angular acceleration of the platform $I_\text{Pl}$, $\dot{\omega}_\text{Pl}$ and of the RW $I_\text{RW}$, $\dot{\omega}_\text{RW}$ are substituted for the corresponding torques, the following formula can be derived:

\begin{equation}
I_\text{Pl} \cdot \dot{\omega}_\text{Pl} = - I_\text{RW} \cdot \dot{\omega}_\text{RW} .
\label{eqn:torque3}
\end{equation}


In \autoref{fig:rotation} an illustration of a platform and a torque generating actuator mounted onto it. The corresponding angular accelerations and MoI are visualized as well as the angular momentum of the system $\vec{L}$, which according to the \textit{Law of Conservation of Angular Momentum} does not change without an acting torque. 



\begin{figure}[H]
{\centering
\includegraphics[width=0.6\textwidth]{Pictures/rotation}
\caption{Illustration of a platform and a torque generating actuator mounted onto it. The MoI and the angular acceleration of the platform $I_\text{Pl}$, $\dot{\omega}_\text{Pl}$ and of the RW $I_\text{RW}$, $\dot{\omega}_\text{RW}$ are visualized as well as the angular momentum vector $\vec{L}$ of the system.}
\label{fig:rotation}
}
\end{figure}


\subsection{Gravitation}
As the influence of the Earths gravitational field cannot be inhibited while inside of its range of influence, the gravitational force of the Earth must be taken into consideration.
This applies in particular to space environment simulations based on Earth, where the acceleration of gravity impacts the test-bed.
In order to prevent potential error torque sources, the test-bed must be aligned orthogonally to the Earth's acceleration of gravity $g_\text{Earth}$.\\

The definition of gravitation and the caused acceleration of gravity on Earth is crucial to determine the required lift force $F_\text{Lift}$ of the air bearing. When considering bearings and especially frictionless bearings, a reference to the gravitational acceleration of the Earth is indispensable since the mass of any body on Earth is directly proportional to its mass and the gravitational acceleration. The mass force, on the other hand, has a direct influence on the acting friction. All design and calculation relevant physical parameters are shown in \autoref{tab:Gravitation}.


\begin{table}[H]
\centering
\resizebox{\textwidth}{!}{%
\begin{tabular}{|l|l|l|l|}
\hline
\textbf{Parameter}                        & \multicolumn{1}{c|}{\textbf{Symbol}} & \multicolumn{1}{c|}{\textbf{Value}} & \multicolumn{1}{c|}{\textbf{Unit}} \\ \hline
\textbf{Normalized Earth Acceleration} & $g_\text{Earth}$ & $9.81$                 & $\frac{m}{s^2}$ \\ \hline
\textbf{Newtonian Gravitational Constant} & $\gamma$                                  & $6.67384 \times10^{-11}$            & $\frac{m^3}{\text{kg} \cdot s^2}$  \\ \hline
\textbf{Radius of the Earth}           & $R_\text{Earth}$ & $6.378 \times10^{6}$   & $m$             \\ \hline
\textbf{Mass of the Earth}             & $M_\text{Earth}$ & $5.9736 \times10^{24}$ & kg              \\ \hline
\end{tabular}%
}
\caption{Physical parameters of the Earth environment \cite{Messerschmid.2017}.}
\label{tab:Gravitation}
\end{table}


Newtons \textit{Law of Gravitation} describes the principles of gravitational fields and the gravitational attraction between two masses. The gravitational force $ \vec{F}_{m} $ acts between two bodies with the masses $m$ and $M$ with the point of mass distance $\vec{r}$. The proportionally factor $\gamma$ is defined as the \textit{Newtonian Gravitational Constant}. The minus sign indicates that the force is attractive \cite{Demtroder.2017}:

\begin{equation}
\vec{F}_{m} = -\gamma \cdot \frac{M \cdot m}{r^{2}} \ \vec{r}.
\label{eqn:LawOfGravity}
\end{equation}

With this Equation the gravitational force $F_\text{m}(r)$ acting on a body with the mass $m$ at the surface of the Earth with the mass $M_\text{Earth}$ in dependence of the radius of the Earth $\vec{r}$ can be calculated:

\begin{equation}
F_\text{m}(r) = -\gamma \cdot \frac{m \cdot M_\text{Earth}}{r^{2}} \ \vec{r}.
\label{eqn:LawOfGravity2}
\end{equation}

If the averaged radius of the Earth $R_\text{Earth}$ is taken into consideration, the acceleration of gravity $g_\text{0}$ acting on the body m can be calculated \cite{Messerschmid.2017}:

\begin{equation}
g_0 = \frac{F_\text{m}(r)}{m} = \frac{-\gamma \cdot M_\text{Earth}}{R_\text{Earth}^{2}} = 9.801 \ \frac{m}{s^{2}}.
\label{eqn:LawOfGravity3}
\end{equation}



This calculation neglects the rotation of the Earth and simplifies the Earth as a spherically symmetrical globe with a constant density and mass distribution. These implicit assumptions made in this calculation lead to an uncertainty. In fact, the acceleration of gravity varies depending on the location and especially on the latitude $\beta$.

This variation along the latitude can be shown through experimental determinations of the Earth acceleration \cite{Demtroder.2017}:

\begin{equation}
\begin{gathered}
\text{Poles} \ : \ g_\text{0}(\beta = 90 ^\circ) = 9.832 \ \frac{m}{s^{2}}. \\
\text{Equator} \ : \ g_\text{0}(\beta = 0 ^\circ) = 9.780 \ \frac{m}{s^{2}}.
\end{gathered}
\label{eqn:LawOfGravity5}
\end{equation}

Through these empirical results of the Earth acceleration an Equation depending on the latitude can be derived \cite{Demtroder.2017}:

\begin{equation}
g_\text{0}(\beta) \approx g_\text{0}(\beta = 0 \ ^\circ) \ (1 + 0.0053024 \cdot \sin \beta^{2} - 5.8 \times10^{6} \cdot \sin 2\beta^{2}).
\label{eqn:LawOfGravity55}
\end{equation}

Generally, the international standardized mean value of the Earth's acceleration $g_\text{Earth}$ is used for the vast majority of mechanical problems. Thus, all calculation in this thesis are made with $g_\text{Earth}$.
The directive of the \textit{European Cooperation in Legal Metrology} proves that these assumptions are justifiably for Germany \cite{EuropeanCooperationinLegalMetrologyWELMEC.15.06.1999}.

\begin{equation}
g_\text{Earth} = \bar{g_\text{0}}(\beta) = 9.80665 \ \frac{m}{s^{2}} \approx 9.81 \ \frac{m}{s^{2}} .
\label{eqn:LawOfGravity6}
\end{equation}

With this standardized value $g_\text{Earth}$ the gravitational force $F_\text{g}$ acting on a mass $m$ on the surface of the Earth can be calculated.

\begin{equation}
F_\text{g} = m \cdot g_\text{Earth}.
\label{eqn:GravForce}
\end{equation}

With this Equation the required lift force $F_\text{Lift}$ of the air bearing to support the required load capacity $W$ can be determined.


\section{Bearing Tribology}
The principles of tribology\footnote{Tribology is the study of interacting surfaces in relative motion and therefore the science and mechanical engineering of friction generated between to moving objects \cite{Qiu.2017}.} and especially bearing\footnote{Bearings are machine parts, used to support dynamic loads and reduce friction and its negative impacts of machinery  \cite{Qiu.2017}.} tribology are introduced here to explain the effects of friction and the friction-reducing functionality of bearings, which is mandatory for this test-bed.\\
 
In 1966, the British engineer H. Peter Jost released his publication \textit{Lubrication (Tribology) Education and Research: A Report on the Present Position and Industry`s Needs}, defining the study of tribology\footnote{Tribology is derived from the ancient Greek word \textit{Tribos} which means friction.}. This definition revolutionized the modern mechanical-engineering-world because to this point tribology had been largely ignored due to its complex nature. Therefore, a great deal of money, material and maintenance time had been wasted through unnecessary consequences of excessive friction and its impacts: Heat and wear.
In the report tribology is defined as \textit{“The science and technology of interacting surfaces in relative motion – and of associated subjects and practices”} \cite{Jost.1990}.\\
Tribology can be described as the scientific deduction and the engineering approach to characterize the interaction between opposing surfaces regarding their relative motion. It is a highly interdisciplinary academic filed combining the scientific disciplines of mathematics, physics, chemistry, mechanics, thermodynamics, aerodynamics, material science and engineering. The three main principles of Tribology are: Friction, Wear and Lubrication \cite{Qiu.2017}.


\subsection{Friction, Wear and Lubrication}
\textit{Friction} is often a negative and important design aspect in every machinery. It leads to loss of power, heat generation and can cause wear, which therefore can lead to failure or damage of the affected machine parts. Furthermore, concerning this thesis, it has the main disadvantage that the occurring friction falsifies the measurement results of the validation of an actuator, which is designed for the space environment.
It can be defined as the resistance force generated by the interaction of two objects moving in relative motion. 
It can be divided into several types depending on the physical state of the two objects. The three most important types for mechanical engineering are:

\newpage 

\begin{itemize}
\setlength{\itemsep}{2pt}
	\item solid – solid friction
	\item solid – liquid friction
	\item solid – gas friction
\end{itemize}

Whereas solid - solid friction usually generates the most friction force at low velocities. Solid - gas friction generates the lowest amount of friction.
During this relative motion of two surfaces friction converts a part of their kinetic energy into thermal energy which generates heat as a byproduct. Another constant byproduct of friction is wear, the extent depends on involved materials and loads.\\\textit{Wear} is the process of surface deformation, damage and gradual removal caused by the friction force. The most important wear mechanisms are: Adhesive wear, Abrasive wear, Fatigue wear, Corrosion wear, Fretting wear, Erosion wear and Cavitation wear.\\

In contradiction \textit{lubrication} is used to reduce friction and its byproducts between two surfaces in relative motion. Lubrication can be described as the intended use of a lubricant\footnote{A substance used to reduce friction, wear, and heat.}. Often liquid or gas lubricants are used as they often produce the lowest friction but solid lubricants exist as well. Suitable lubrication can significantly increase bearing life and reliability. There are several lubrication mechanisms such as: Hydrodynamic, Hydrostatic, Aerodynamic, Aerostatic, Mixed and Solid Lubrication. The mechanical engineering approach for using this lubrication effect are bearings.\\
\textit{Bearings} are a key part in a significant majority of machines in order to reduce friction between moving components. They are also responsible for supporting dynamic loads and constraining relative motion to a desired DOF for the moving segments. There are several bearing categories but for a fundamental classification bearings can be divided into two separate groups: Rolling bearings and Sliding bearings. This classification based on the difference of the allowed relative motion form. Bearing tribology is a main aspect of the study of tribology and includes lubrication, friction, heat and wear regarding bearings \cite{Qiu.2017}\cite{Brecher.2017}.


\section{Air Bearing}
\subsection{Air Bearing Overview}
\label{sec:Airbearingoverview}
Aerostatic bearings utilizes filtered air as a lubricant. As they do not contain any rolling parts, they can be classified under the category of sliding bearings. In addition, their aerostatic lubrication enables the frictional state of the two surfaces to change in relative motion. Thus, a solid-solid friction case without bearings, which usually generates a high frictional force, becomes a solid-gas friction, which generates a low frictional force, depending on the materials \cite{Qiu.2017}. \\
Several physical phenomena can be taken into account for providing such a lubricant effect, but this thesis is limited to external pressurization which leads to an aerostatic lubrication. Externally pressurized air bearings are referred to as hydrostatic as they do not require any relative motion of the bearing to produce lubrication. The lubrication is solely based on aerodynamic principles of the pressurized air flowing through the bearing and creating a fluid film through the pressure force by doing so.
The continuous supply of compressed air within the bearing creates a fluid film lubrication between the opposing bearing sides of the stationary and moving part. The appropriate air pressure and mass flow of the pressurized air are crucial for the performance of the bearing \cite{Bartz.2014}\cite{Slocum.1992}.
This thin film of air separates the moving objects from each other by only a few $\mu$m, resulting in the elimination of almost all friction. In fact, the normal operating range of the fluid film thickness $h_\text{Fluid}$  in conventional air bearings is between $5- 20 \ \mu$m, however bearings with thinner or larger clearances function as well. This extreme thinness of the fluid film requires the surfaces of an air bearing to be machined with high precision. Surfaces with an average surface roughness $Ra \leqq 5 \ \mu m$ are recommended. In addition, the bearing must be protected from contamination, since dust particles of the order of $\mu$m are already present in air and can therefore cause undesirable friction effects \cite{Hamrock.1991}\cite{Gao.2019}.
There are high functioning bearings applicable for precision applications and provide nearly frictionless bearing effects. This is possible through the extremely low viscosity of air. In fact, the viscosity of air is $1000$ times lower than even the thinnest mineral oils used in conventional oil lubricated bearings. This leads to a near elimination of the viscous resistance and therefore friction forces inside the bearing. Since for an ideal air bearing only the very low air friction or air resistance remains as part of the friction, they can be considered as nearly frictionless. This non-contact bearing effect provides a solution to the traditional bearing related problems friction, wear and heat. When applied to rotational motion this nearly frictionless attribute can be transferred to as nearly torque free. The most important strengths and weaknesses of air bearings are listed in \autoref{tab:airbearingadv} \cite{Hamrock.1991}\cite{Gao.2019}\cite{NewWayAirBearingsInc..2006}\cite{Wang.1993}.\\


\begin{table}[H]
\centering
\resizebox{\textwidth}{!}{%
\begin{tabular}{|l|l|}
\hline
\multicolumn{1}{|c|}{\textbf{Advantages}}        & \multicolumn{1}{c|}{\textbf{Disadvantages}}                                                                            \\ \hline
\begin{tabular}[c]{@{}l@{}}1. Nearly frictionless due to extremley \\ low friction or viscous resistance.\end{tabular} &
  \begin{tabular}[c]{@{}l@{}}1. Really weak load capacity compared \\ to oil lubricated bearings.\end{tabular} \\ \hline
2. Filtered Air is a simple and clean lubricant. & \begin{tabular}[c]{@{}l@{}}2. Bearing surfaces require extremly fine \\ finish and low surface roughness.\end{tabular} \\ \hline
3. Filtered Air does not contaminate surfaces.   & \begin{tabular}[c]{@{}l@{}}3. Bearing part alignment mut be extremley \\ good accurate.\end{tabular}                   \\ \hline
\begin{tabular}[c]{@{}l@{}}4. Operates well from extremely low \\ to extremely high temperatures.\end{tabular} &
  \begin{tabular}[c]{@{}l@{}}4. Bearing tolerances  and clerances \\ must be extremley accurate.\end{tabular} \\ \hline
5. Extremley high load capacity at low speeds.   & 5. Components availability                                                                                             \\ \hline
\end{tabular}%
}
\caption{Listing of the main advantages and disadvantages of air bearings.}
\label{tab:airbearingadv}
\end{table}



The fluid film lubrication is achieved by supplying air into the bearing gap and separating the two bearing surfaces. The supplied air then discharges to the surrounding environment from the exit edges of the bearing clearance as visualized in \autoref{fig:airbearing}. The illustration shows an aerostatic thrust bearing pad with a pocket restrictor. Compressed air is fed through the inlet with the externally supplied pressure $p_\text{s}$. As the air passes through the orifice at the inlet the pressure $p_\text{s}$ drops to $p_\text{d}$ in consequence of the geometrical change and is then radiating to the edge of the air film where it escapes into the environment and the pressure drops to $p_\text{a}$. This pressure distribution along the air film is visualized beneath through a pressure diagram. Along the radius from the orifice exit to the edge of the bearings air gap the pressure gradually decreases. The load capacity of the bearing can be described as the integral of the pressure curve over the radius and corresponds exactly to the load of the moving bearing part $F_\text{W}=W$. For this load, the fluid film thickness is $h_\text{0}$. If the bearing is now loaded with an additional load F: $F'_\text{W}=W+F$, the clearance is reduced to $h_\text{Fluid}$. A smaller clearance, however, results in a geometrical change in the air gap, which causes a change in the pressure distribution along the radius. The pressure at the outlet of the orifice increases to $p'_\text{d}$, which affects the pressure distribution, thus increasing the load capacity of the bearing. This procedure shows how the pressure distribution and load capacity of the bearing changes depending on the fluid film thickness $h_\text{Fluid}$. Of course, there is also a dependence on the bearing area, as this is calculated integrally over this area. These changes also affect the stiffness of the bearing, as it depends mainly on the load capacity in relation to the fluid film thickness and thus on the gradient of the pressure distribution. The most uniform pressure distribution possible provides the highest possible stiffness for a bearing \cite{Hamrock.1991}\cite{Gao.2019}\cite{NewWayAirBearingsInc..2006}.


\begin{figure}[H]
{\centering
\includegraphics[width=1.\textwidth]{Pictures/airbearing}
\caption{Illustration of an externally pressurized air bearing pad \cite{Gao.2019}.}
\label{fig:airbearing}
}
\end{figure}


In general, this is accomplished through a specific restrictor, which like the name suggests restricts and therefore regulates the air flow inside the bearing gap. As this restriction heavily influences the characteristics, special attention should be given to its selection. The five conventional restrictor types as well as a comparison of their relevant performance properties is given in \autoref{fig:restrictor}.\\
The most commonly used restrictor category are the orifice restrictors as they are easy to manufacture and plenty of design information is available. They can be classified as turbulent flow devices as they create turbulent air flow inside the orifice. Furthermore, they can be divided into annular or inherently orifice and simple or pocketed orifice.\\
As a next category slot restrictors can be mentioned as they are quite similar to orifice restrictors. They only differ in the width of the air inlet. As they generate laminar instead of turbulent flow inside the air film they provide better stability compared to orifice restrictors.\\
When the air inlet is arranged with specially arrayed grooves around the orifice along the bearing surface, they form a groove restrictor. Through the additional groves they provide a more uniform pressure distribution within the air film than a normal orifice restrictor.\\
Porous restrictors form the last and special restrictor category as they are the only type requiring a special material. Through porous material air can be feed inside the clearance. Inside the porous material air can be fed through microscopic pores, which leads to an accumulation of extremely high numbers of extremely small orifice restrictors along the bearing surface. Thus, providing a uniform pressure distribution along the radius of the bearing. In general, the porous medium used for such a restrictor is graphite.\\ 
In conclusion all types of restrictors have strengths and weaknesses, and all would be applicable for this test-bed but the decision was made to use a graphite restrictor for the realization of this test-bed. The main arguments for this decision are the high stiffness, load capacity, damping and pneumatic stability compared to other types of air bearing restrictors \cite{Hamrock.1991}\cite{Gao.2019}\cite{NewWayAirBearingsInc..2006}.

\begin{figure}[H]
{\centering
\includegraphics[width=1.\textwidth]{Pictures/restrictor}
\caption{Illustration of the five most conventional restrictor types and their properties \cite{Gao.2019}.}
\label{fig:restrictor}
}
\end{figure}


\subsection{Fluid Film Lubrication}
\label{sec:lubrication}
Externally pressurized Air bearings function by using pressurized gas\footnote{In this thesis air is the used gas. Furthermore, the air is considered as an ideal gas.} as a fluid lubricant. This fluid film lubrication is governed by the \textit{Laws of Fluid Dynamics}. The following Section provides a substantial overview about these laws \cite{Hamrock.1991}.\\ 


The performance of an air bearing is largely dependent on the pressure distribution in the clearance between the two contact surfaces. The three basic Equations from fluid mechanics and thermodynamics for the calculation of air bearings are listed below.\\The main Equations used to describe the flow within the bearing and thus its performance are the \textit{Navier Stokes Equation}. These are shown in (\ref{eqn:Navier}) in their vector notation. From these the Reynolds Equation can be derived, which is intended to describe the fluid film lubrication. However, this will not be discussed further here, as this thesis is limited to a simplified calculation of the air bearing performance by:

\begin{equation}
\vec{v} \ \nabla \vec{v} + \frac{\delta \vec{v}}{\delta t} = - \nabla \frac{p}{\rho} + \mu \Delta \vec{v} + \vec{f} .
\label{eqn:Navier}
\end{equation}

In addition to the \textit{Navier Stokes Equations}, both the continuity Equation (\ref{eqn:Konti}) and the Equation of state for ideal gases (\ref{eqn:idealesGas}) are of exceptional relevance. The continuity Equation describes that the amount of mass entering a system equals the amount of mass leaving a system plus the accumulation of mass within the system. In this differential form it can be described as the derivative of the fluid density $\rho$ to the time $t$ in combination with the divergence of the fluid density and fluid velocity $\vec{v}$:

\begin{equation}
\nabla(\rho \cdot \vec{v}) + \frac{\delta \rho}{\delta t} = 0 .
\label{eqn:Konti}
\end{equation}

The ideal gas Equation describes the relation between the pressure of the ideal gas $p$ in correspondence to its density $\rho$ and temperature $T$ as well as the ideal gas constant $R$. For air the ideal gas konstant has a value of $R = 287 \ \frac{J}{\text{kg} \cdot K}$:

\begin{equation}
p = R \cdot \rho \cdot T .
\label{eqn:idealesGas}
\end{equation}

For porous media, \textit{Darcy's law} (\ref{eqn:Darcy}) should also be mentioned. \textit{Darcy's Law} can be used to describe the flow through an isentropic porous media if the Reynolds number is very small based on the pore size or particle diameter $d_\text{p}$. The momentum Equation for this flow is shown in (\ref{eqn:Darcy}), where $p$ is the pressure within the medium, $\mu$ is the viscosity of the fluid and $v$ is the velocity. Here $v$ is defined as a superficial velocity, because the medium is considered to be a continuum, which means that the details of the pore structure are not considered. The proportionality factor between the pressure $p$ and the velocity $v$ is defined as the permeability of the media $K$, which, as described in (\ref{eqn:Permeability}), depends on the porosity of the medium $\Phi$ and a constant to parameterize the microscopic geometry of porous materials $a$ \cite{Vafai.2005}:

\begin{equation}
- \nabla p = \frac{\mu \cdot v}{K} .
\label{eqn:Darcy}
\end{equation}


\begin{equation}
K = \frac{\Phi^{3} d_\text{p}^{2}}{a(1- \Phi^{2})} .
\label{eqn:Permeability}
\end{equation}

However, if the permeability $K$ of a graphite body with a height $H$ is known, it can be used to compute the flow velocity $v$ as shown in Equation (\ref{eqn:flowv})\cite{Plante.2005}:

\begin{equation}
v = \frac{K(\Delta p^{2})}{2 \mu Hp_\text{s}} .
\label{eqn:flowv}
\end{equation}

Furthermore, the assumption of a purely axial and locally uniform flow over small bearing surfaces in the porous media can be made. This allows the mass flow rate $ \dot{m} $ to be calculated in a general porous element with a cross section area $A_\text{P}$ \cite{Plante.2005}:

\begin{equation}
\dot{m} = \rho  v A_\text{P} .
\label{eqn:massf}
\end{equation}

If inserting (\ref{eqn:flowv}) and (\ref{eqn:idealesGas}) in (\ref{eqn:massf}) the following dependence of the mass flow rate is obtained:

\begin{equation}
\dot{m} = \frac{A_\text{P} K \Delta p^{2} }{2 \mu R T H} .
\label{eqn:massf2}
\end{equation}


The analytical solution for the calculation of an air bearing used in this thesis is based on a simplified version of the\textit{ Navier Stokes Equations}. Since the solution of these Equations can be very complex, a 1D generalized flow theory is used, which drastically simplifies the usage case. The simplified theory describes the flow as a function of the derivative of the Mach number $ M $ except for singularities at $M = 0$ or $M = 1$ by equating it with selectable relevant flow effects. Depending on the application, any combination of predominant flow effects can be combined.\\
For the application of simulating an air bearing, the change in surface area, as well as the mass flow of the air and the air friction are decisive. All other influences are negligible. 
Therefore the flow can be described with the following Equation \cite{Plante.2005}:


\begin{equation}
\frac{ \Delta (M^{2})}{M^{2}} = \frac{2 \Delta M}{M} = C_\text{1}\frac{ \Delta A}{A} + C_\text{2} 4 f_\text{c} \frac{ \Delta r}{D_\text{h}} + C_\text{3}\dfrac{ \Delta \dot m}{\dot m} .
\label{eqn:1D}
\end{equation}


Whereas the mentioned coefficients can be described by the following Equations:


\begin{equation}
C_\text{1} = - \frac{2(1 + \frac{K - 1}{2} M^{2})}{1 - M^{2}} .
\label{eqn:1D_C1}
\end{equation}

\begin{equation}
C_\text{2} = \frac{K M^{2} ( 1 + \frac{K - 1}{2} M^{2})}{1 - M^{2}} .
\label{eqn:1D_C2}
\end{equation}

\begin{equation}
C_\text{3} = \frac{2 ( 1 + K M^{2}) (1 + \frac{K - 1}{2} M^{2})}{1 - M^{2}} .
\label{eqn:1D_C3}
\end{equation}

The compressible friction coefficient of the 1D flow theory $f_\text{c}$ can be estimated using the hydraulic Darcy coefficient factor for incompressible pipe flows $f_\text{h}$. This hydraulic friction coefficient can be approximated in relation to the hydraulic Reynolds number $R_\text{H}$ as described in (\ref{eqn:hfric}) \cite{Plante.2005}:

\begin{equation}
f_\text{c}=\frac{1}{4} f_\text{h} = \frac{1}{4} \frac{96}{R_\text{H}} = \dfrac{24 \mu}{\rho v D_\text{h}} .
\label{eqn:hfric}
\end{equation}

Thus, the 1D general flow theory can be used to simulate the load capacity of the bearing $W$ as a function of the pressure distribution $P$ within the clearance over the radius $R$. The load capacity relates to the possible lift Force $F_\text{Lift}$ of the bearing and the acceleration of gravity on Earth $ g_\text{Earth}$ as $F_\text{Lift} = W \cdot g_\text{Earth}$. The lift Force $F_\text{Lift}$ depends on the fluid film thickness $h_\text{Fluid}$ . Here, the proportionality factor of the two values forms the stiffness $S$ of the bearing, which is an important criterion for the design of an air bearing, as mentioned above \cite{Cui.2018}:

\begin{equation}
S = \frac{\Delta F_\text{Lift}}{\Delta h} .
\label{eqn:stiff}
\end{equation}


Furthermore a rough estimation of the lift force $F_\text{Lift}$ can be made through the bearing efficiency. This efficiency\footnote{The air bearing efficiency indicates how much of the pressure force generated by the compressed air can be converted into lift force.} can be approximated and was verified to be $\eta = 30 \ \%$ \cite{Plante.2005}:

\begin{equation}
F_\text{Lift} =(P_\text{s} - P_\text{a})A_\text{p} \eta .
\label{eqn:coarsef}
\end{equation}

If a lift force $F_\text{Lift}$ has been defined, it can be used to determine the load capacity $W$ that the bearing can support. To support the weight of a body with the mass $m$, the bearing muss be able to provide a lift Force higher or equal than the gravitational force $F_\text{g}$ of the body:

\begin{equation}
F_\text{Lift} = W \cdot g_\text{Earth} \geq F_\text{g} = m \cdot g_\text{Earth} .
\label{eqn:LawOfGravity7}
\end{equation}






\subsection{Pressure Plenum Chamber}
\label{sec:plenumchamber}
A plenum chamber is an enclosed space in which air, or any other gas or liquid is under plenum condition. Meaning the air pressure in the enclosed chamber is greater than the outside atmosphere which leads to a higher plenum chamber pressure and a pressure gradient between the enclosed plenum space and the environment. Air is forced into the chamber for slow distribution through ducts and restrictors \cite{.2003}.\\
The resulting pressure gradient impacts the chamber walls leading to a pressure force effecting the material. In order to prove whether the chamber walls can withstand this pressure force, a strength calculation is necessary.\\
To determine the possible internal pressure a cylindrical body can withstand without deformation or damage dependent on its dimensions and the strength of its material Barlow's formula can be used \cite{.2003}.\\
The computing of this strength is valid for all approximately rotational-symmetric cylindrical body like pipes, cylinders and cylindrical plenum chambers with a relative small wall thickness and the detailed steps required for this calculation can be found in the DIN EN 13480-3 \cite{.NormDINEN13480}.\\

A summarized and simplified calculation procedure is proposed by H.Dietmann \cite{Dietmann.1992}.\\
The tangential stress $\sigma_\text{t}$ inside the plenum chamber can be calculated as shown in Equation (\ref{eqn:Kessel1}) in dependency of its inner pressure $p_\text{i}$ the chamber wall thickness $s$ and the inner plenum diameter $d_\text{i}$:

\begin{equation}
\sigma_\text{t} = \frac{p_\text{i} \cdot d_\text{i}}{2 \cdot s} .
\label{eqn:Kessel1}
\end{equation}
 
 
Using the same parameters the axial stress $\sigma_\text{a}$ inside the plenum chamber can be calculated as shown in (\ref{eqn:Kessel2}):

\begin{equation}
\sigma_\text{a} = \frac{p_\text{i} \cdot d_\text{i}}{4 \cdot s} .
\label{eqn:Kessel2}
\end{equation}

Furthermore, the inner pressure $p_\text{i}$ causes a radial stress on the inside surface area which is directly opposite with the same value $\sigma_\text{r} = - p_\text{i} $. On the stress free outer surface, the radial stress is $\sigma_\text{r} = 0 $. For thin walled plenum chambers usually a constant average value is considered for describing the radial stress as shown in Equation (\ref{eqn:Kessel3}):

\begin{equation}
\sigma_\text{r} = - \frac{p_\text{i} \cdot d_\text{m}}{2} .
\label{eqn:Kessel3}
\end{equation}

In addition, the inner diameter and the mean diameter can be assumed to be equal $ d_\text{i} \approx  d_\text{m} $ for thin-walled applications.\\
In \autoref{fig:Kesselformel} the stress components are shown through various sectional views. On the left side a free cut of a plenum chamber piece can be seen. On the right side, a longitudinal section of a plenum chamber is shown at the top and a cross section at the bottom.

\begin{figure}[H]
{\centering
\includegraphics[width=0.7\textwidth]{Pictures/Kesselformel}
\caption{Schematic overview of the stress components inside a thin walled plenum chamber. The internal pressure $p_\text{i}$ and the resulting stresses $\sigma_\text{r}$ , $\sigma_\text{t}$ and $\sigma_\text{a}$ can be seen. Also the inner diameter $d_\text{i}$ and the wall thickness $s$ of the plenum chamber are visualized \cite{Lapple.2016}.}
\label{fig:Kesselformel}
}
\end{figure}

The comparison of the two stress Equations demonstrates that the tangential stress is greater than the axial stress by a factor of $2$.
For a stress analysis to prove the strength of the plenum chamber the triaxial state of stress must be reduced to a uniaxial fictive state of stress through strength hypotheses. For ductile materials\footnote{Ductility describes the ability of a material to undergo significant plastic deformation before destruction.}, which are commonly used for plenum chambers the maximum shear stress hypothesis offers the most conservative solution and is therefore generally used as the method to determine an uniaxial equivalent comparison stress $\sigma_\text{c}$ as shown in Equation (\ref{eqn:Kessel5}):

\begin{equation}
\sigma_\text{c} = \sigma_\text{t} - \sigma_\text{r} = \frac{p_\text{i} \cdot d_\text{m}}{2 \cdot s} + \frac{p_\text{i} \cdot d_\text{m}}{2} = \frac{p_\text{i} \cdot d_\text{m}}{2 \cdot s} .
\label{eqn:Kessel5}
\end{equation}

For this the average plenum diameter can be described  using the chamber wall thickness $s$, the inner wall diameter $d_\text{i}$ and the outer wall diameter $D_\text{o}$ as shown in Equation (\ref{eqn:Kessel4}):

\begin{equation}
d_\text{m} = d_\text{i} + s = \frac{D_\text{o} + d_\text{i}}{2} .
\label{eqn:Kessel4}
\end{equation}

For all calculations used to determine the stress caused by a pressure gradient between the inner plenum chamber pressure and the outer pressure the standardized mean value of the atmospheric pressure at sea level can be used \cite{DeutschesInsitutfurNormunge.V.DIN.}:

\begin{equation}
p_\text{a} = 101325 \ \text{Pa} \approx 1.013 \ \text{bar} .
\label{eqn:Kessel6}
\end{equation}

Through this approach the required dimensions for a plenum chamber for a given inner pressure can be computed using Barlow's formula. If the calculated equivalent stress is lower than the stress the material can withstand according to the material data, than the plenum chamber is correctly designed. The material data can usually be taken from the material data sheets. Note that the dimensions should be strengthened through an appropriate safety margins.



\section{Materials}
As multiple materials are used and compared to each other in the curse of developing a suitable design they should be discussed in detail. Therefore, all relevant material properties are listed below in \autoref{tab:mats}. PLA (\textit{Polylactide}) is used for all 3D printed parts. The aluminum alloy \textit{AlMg3} is used for various components externally manufactured by \textit{Laserhub GmbH} and referred to as aluminum in this thesis. \textit{V2A} and \textit{16MnCr5} are high performance steel alloys which are used for some components externally manufactured by \textit{Facturee}. All values are referenced from the corresponding material data sheets of the manufacturers or table books \cite{Gomeringer.2014}.


\begin{table}[H]
\centering
\resizebox{\textwidth}{!}{%
\begin{tabular}{|c|c|c|c|c|c|}
\hline
\textbf{Material} &
  \textbf{\begin{tabular}[c]{@{}c@{}}V2A\\  (X5CrNi18-10)\\ \cite{FactureecwmkGmbH.09.2011}\cite{Gomeringer.2014}\end{tabular}} &
  \textbf{\begin{tabular}[c]{@{}c@{}}AlMg3\\ \cite{LaserhubGmbH.10.2019}\cite{Gomeringer.2014}\end{tabular}} &
  \textbf{\begin{tabular}[c]{@{}c@{}}16MnCr5\\ \cite{FactureecwmkGmbH.06.2020}\cite{Gomeringer.2014}\end{tabular}} &
  \textbf{\begin{tabular}[c]{@{}c@{}}PLA\\ \cite{Filamentworld.2020}\end{tabular}} &
  \textbf{Units} \\ \hline
\textbf{Material Number}      & 1.4301 & 3.3535 & 1.7131 & -    &                         \\ \hline
\textbf{Density $\rho$}              & 7900   & 2660   & 7810   & 1240 & $\frac{\text{kg}}{\text{m}^{3}}$ \\ \hline
\textbf{Tensile Strength $R_\text{m}$}  & 680    & 215    & 1080   & 110  & MPa                     \\ \hline
\textbf{Yield Strength $R_\text{p0.2}$} & 210    & 90     & 590    & -    & MPa                     \\ \hline
\end{tabular}%
}
\caption{Table including all manufacturing materials and their relevant material properties.}
\label{tab:mats}
\end{table}