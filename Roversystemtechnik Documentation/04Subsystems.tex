\chapter{Subsystems}
\label{chap:subsystems}
.....



\section{Rover}
\label{sec:rover}
...
\section{Structure and Mechanics}
\label{sec:mechanics}
...
\section{Communications and Command and Data-Handling}
\label{sec:comm}
...
\section{Payload}
\label{sec:payload}
...
\section{Thermal Control}
\label{sec:thermalcontrol}
...
\section{Electrical Power System}
\label{sec:EPS}
The EPS (Electrical Power System) is the subsystem responsible for the electrical power supply of INSPIRE. It consists of four funadmental parts, which are the energy source, the PCDU unit (Power Control and Distribution)and the Energy Storage as well as the rover subsystems as the consumers.

\subsection{EPS Budget and Overview}

\begin{figure}[htb]
{\centering
\includegraphics[width=0.7\textwidth]{Media/epsflowchart}
\caption{Functional Flow Chart for the EPS Subsystem}
\label{fig:epsflowchart}
}
\end{figure}

\subsection{EPS Power Control and Distribution}

\subsection{Energy Source}

\subsection{Energy Storage} 



\clearpage

\section{Radiation}
\label{sec:Radiation}

Compared to the radiation environment near Earth the radiation environment near Jupiter is multiple times stronger. It has the highest radiation levels of any planet in our solar systems \cite{Platzhalter}. In order to survive these harsh environmental conditions, special emphasis must be placed on the radiation protection. In \autoref{fig:trappedprotonelectronfluxes}, the average trapped proton and electron fluxes on Europa's orbit around Jupiter are shown in comparison to the outer Van Allen radiation belt around Earth. However, in contrast to the Van Allen radiation belt, the duration within the radiation environment on Europa cannot be minimised and the rover has to be designed to withstand the entire mission duration of 30 days.

\begin{figure}[htb]
     \centering
     \begin{subfigure}[b]{0.475\textwidth}
         \centering
         \includegraphics[width=\textwidth]{Media/E_Electron_Flux}
         \caption{Average spectra of trapped electrons around Earth.}
         \label{fig:trappedelectronsEarth}
     \end{subfigure}
     \hfill
     \begin{subfigure}[b]{0.475\textwidth}
         \centering
         \includegraphics[width=\textwidth]{Media/E_Proton_Flux}
         \caption{Average spectra of trapped protons around Earth}
         \label{fig:trappedprotonsEarth}
     \end{subfigure}
     \hfill
     \begin{subfigure}[b]{0.475\textwidth}
         \centering
         \includegraphics[width=\textwidth]{Media/J_Electron_Flux}
         \caption{Average spectra of trapped electrons around Jupiter}
         \label{fig:trappedelectronsJupiter}
     \end{subfigure}
     \hfill
     \begin{subfigure}[b]{0.475\textwidth}
         \centering
         \includegraphics[width=\textwidth]{Media/J_Proton_Flux}
         \caption{Average spectra of trapped protons around Jupiter}
         \label{fig:trappedprotonsJupiter}
     \end{subfigure}
        \caption{Average trapped proton and electron fluxes on an orbit around earth at 25,000 km, through the outer Van Allen radiation belt, and on Europa's orbit around Jupiter.}
        \label{fig:trappedprotonelectronfluxes}
\end{figure}

In oder to design and evaluate different radiation protection approaches, different calculations have to be performed. For this purpose the ESA SPace ENVironment Information System (SPENVIS) is used \cite{Platzhalter}. All calculations and figures in \autoref{sec:Radiation} are performed with SPENVIS unless otherwise stated.

\cleardoublepage