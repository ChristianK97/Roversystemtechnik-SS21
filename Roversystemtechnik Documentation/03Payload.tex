
\chapter{Payload}
\label{chap:payload}

\section{Sterovision Camera / Observation / Perception}

The INSPIRE rover is equipped with five individual cameras. Two are used as stereo vision cameras on an hight adjustable and rotatable telescope arm on the front side of the rover. This ist used to capture a detailed 3D model of the environment with which sizes and distances can be estimated. The remaining three cameras are used as has-cameras which are necessary to obtain data regarding the nearby environment. All cameras are equipped with radiation hardened lenses to prevent browning of the lenses. The main tasks of the camera system is the provision of scientific data and navigation related data. More details regarding the navigation and autonomy are provided in \autoref{sec:ControlandAutonomy}.

\section{Ground RADAR}

\section{Ice Core Drill}


\section{RadHard Solar Arrays}
\label{subsec:radhard}
As a secondary mission goal for INSPIRE a cooperation with the european project RadHard which is led by the german solar array manufacturer Azure Space is intended. They are currently developing a new generation of $4$ Juniction solar cells with an efficiency of up to $35 \% $. But the main feature of the new solar arrays is their radiation hardness which will be the highest radiation hardness ever designed with an efficiency of $>3 \% $ after $1E15 \ cm^{-2} \ 1MeV$ electron irradiation. So the Jupiter environment with its extreme radiation would be the best suitable destination for a test and evaulation mission of this new technology. Therefore INSPIRE will be equipped with $8$ RadHard solar cells with a total surface area of $0.0248 m^2$ for a technology demonstration\cite{FraunhoferInstituteforSolarEnergySystemsISE.2021}.

