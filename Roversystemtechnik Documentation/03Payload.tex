
\chapter{Payload}
\label{chap:payload}

The Main Payload (Ground Radar and Ice Core Drill) as well as the secondary payloads (Cam Head and Rad Hard Solar cells) will be explained in Detail in this Chapter.

\section{Ground RADAR}
The Ground Radars main task is the identification of suitable drilling sites. Additionally every radar campaign will contribute to further understanding of the ground composition on Europa.\\

due to its small dimensions the CRUX GPR is selected . The system is tested for lunar application at 800 MHz resulting in a resolution of 15 cm and a penetration depth of 5 m [CRUX RADAR REF]. The INSPIRE mission makes use of a 1.5 GHz frequency to increase the resolution. \\
Reduced penetration depths are acceptable since the depth of interest for the ice core sampling is 10 cm. \\

Additionally high frequencies lead to compact antenna design which is beneficial to the INSPIRE mission due to weight constraints. 
Based on [Paper \& Website reference] a custom patch antenna with the properties in \autoref{tab:GPR-A-Prp} is proposed.

\begin{table}[h]
\centering
\begin{tabular}{lllll}
Substrate ${\varepsilon}_{r}$ & Width & Length & Height & Mass   \\ \hline\hline
20                         & 30 mm & 20 mm  & 2 mm   & 2,73 g  \\ \hline
\end{tabular}
\caption{GPR antenna properties}
\label{tab:GPR-A-Prp}
\end{table}

\section{Ice Core Drill and APXS Sensor}

The ice core drill is an in-house development which is based on the NanoDrill from Honeycomp and the drill from Philae. The drill is made of titanium to ensure that it does not deform or even break through during the drilling process.
The ice core sample that can be obtained has a length of 100mm and a diameter of 10mm
In order to save space, the drill is folded in while driving and is then unfolded in the "Payload: IceCoreDrill" mode, which is illustrated in the following figures. (PICTURE)
When the drilling process is finished, it is folded in again and the trap doors are closed.
Now the sample can be pushed out with the help of a plunger and meanwhile it can be analyzed by the APXS sensor. (Figure XXX)
When the sample has left the drill body and the analysis is complete, the rover can switch back to Locomotion mode and look for a new place to drill. As soon as a new drilling process is started, the previously taken sample falls to the ground at the same time to make room for a new one.

\section{Sterovision Camera / Observation / Perception}

The INSPIRE rover is equipped with five individual cameras. Two are used as stereo vision cameras on an hight adjustable and rotatable telescope arm on the front side of the rover. This ist used to capture a detailed 3D model of the environment with which sizes and distances can be estimated. The remaining three cameras are used as has-cameras which are necessary to obtain data regarding the nearby environment. All cameras are equipped with radiation hardened lenses to prevent browning of the lenses. The main tasks of the camera system is the provision of scientific data and navigation related data. More details regarding the navigation and autonomy are provided in \autoref{sec:ControlandAutonomy}.

\section{RadHard Solar Arrays}
\label{subsec:radhard}
As a secondary mission goal for INSPIRE a cooperation with the european project RadHard which is led by the german solar array manufacturer Azure Space is intended. They are currently developing a new generation of $4$ Juniction solar cells with an efficiency of up to $35 \% $. But the main feature of the new solar arrays is their radiation hardness which will be the highest radiation hardness ever designed with an efficiency of $>3 \% $ after $1E15 \ cm^{-2} \ 1MeV$ electron irradiation. So the Jupiter environment with its extreme radiation would be the best suitable destination for a test and evaulation mission of this new technology. Therefore INSPIRE will be equipped with $8$ RadHard solar cells with a total surface area of $0.0248 m^2$ for a technology demonstration\cite{FraunhoferInstituteforSolarEnergySystemsISE.2021}.

