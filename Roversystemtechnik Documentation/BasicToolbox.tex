%%%%%%%%%%%%%%%%%%%%%%%%%%%%%%%%%%%%%%%%%%%%%%%%%%%%%%%%%%%%%%%%%%%%%%%%%%%%%%%%%%%%%%%%%%%%%%%%%%%%%%%%%%%%%%%%%%%%%%
%
%                                          TOOL BOX
%
%%%%%%%%%%%%%%%%%%%%%%%%%%%%%%%%%%%%%%%%%%%%%%%%%%%%%%%%%%%%%%%%%%%%%%%%%%%%%%%%%%%%%%%%%%%%%%%%%%%%%%%%%%%%%%%%%%%%%%%

%Generelle Befehle:

Kursiv \textit{TEXT}
Fett \textbf{TEXT}
Standardtext \text{TEXT} (Wichtig innerhalb von Formeln)
Auskommentieren ist möglich mit %
Zeilenumbrüche mit \\
Formeln innerhalb des Fließtexts: $p_\text{c} = p_\text{a}$

% Überschriften und Co

% Kapitelüberschrift z.B. 1.
\chapter{Kapitelüberschrift}
% Labels sind sehr nützlich zum referenzieren und verlinken von Kapiteln (aber auch Bildern etc.)
\label{chap:Kapitel}
Durch ein Label kann ich mich hier nun auf das Kapitel beziehen. Wir betrachten Kapitel \ref{chap:Kapitel}. Noch besser ist es mit \autoref{chap:Kapitel}.

%Unterkapitel Überschrift z.B. 1.1.
\section{Unterkapitel}
\label{sec:unterkapitel}

%Unterunterkapitel z.B. 1.1.1.
\subsection{Unterunterkapitel}
\label{subsec:unterunterkapitel}


% Grafiken, Tabellen und Formeln

% Grafiken

% Einfache Grafik (Jpg, png)

\begin{figure}[H]
{\centering
\includegraphics[width=0.5\textwidth]{Pictures/Bildname}
\caption{Bildbeschreibung}
\label{fig:Bild}
}
\end{figure}

% Oder etwas komplizierter: Wenn man mehrere Grafiken zusammen anordnen will (z.B. Plots)

\begin{figure}[H]
    \centering
    \subfloat[Name1]{{\includegraphics[width=0.33\textwidth]{Pictures/Name1}}}
    \hfill
    \subfloat[Name2]{{\includegraphics[width=0.322\textwidth]{Pictures/Name2}}}
    \hfill
    \subfloat[Name3]{{\includegraphics[width=0.293\textwidth]{Pictures/Name3}}}
    \caption{Bildbeschreibung}
    \label{fig:Bild3}
\end{figure}

% Oder noch komplizierter. Hier wären es 10 Bilder auf einer Seite
\begin{figure}[H]
    \centering
    \subfloat[Semi-Spindle I]{{\includegraphics[width=0.3\textwidth]{Pictures/1DOF1}}}
    \qquad
    \subfloat[Semi-Spindle I]{{\includegraphics[width=0.3\textwidth]{Pictures/1DOF6}}}
    
    \subfloat[Semi-Spindle II]{{\includegraphics[width=0.3\textwidth]{Pictures/1DOF2}}}
    \qquad
    \subfloat[Semi-Spindle II]{{\includegraphics[width=0.3\textwidth]{Pictures/1DOF7}}}

    \subfloat[Semi-Spindle III]{{\includegraphics[width=0.3\textwidth]{Pictures/1DOF3}}}
    \qquad
    \subfloat[Semi-Spindle III]{{\includegraphics[width=0.3\textwidth]{Pictures/1DOF8}}}
   
    \subfloat[Spindle I]{{\includegraphics[width=0.3\textwidth]{Pictures/1DOF4}}}
    \qquad
    \subfloat[Spindle I]{{\includegraphics[width=0.3\textwidth]{Pictures/1DOF9}}}    
    
    \subfloat[Spindle II]{{\includegraphics[width=0.3\textwidth]{Pictures/1DOF5}}}
    \qquad
    \subfloat[Spindle II]{{\includegraphics[width=0.3\textwidth]{Pictures/1DOF10}}}
    \caption{Schematic illustration of 1-DOF designs in CAD. On the left an isometric view of each design is given. On the right a cross section illustration can be seen.}
    \label{fig:1DOF}
\end{figure}



% Auflistungen und Bullet Points

\begin{itemize}
\setlength{\itemsep}{2pt}
	\item Punkt 1
	\item Punkt 2
	\item Punkt 3
	\item Punkt 4
\end{itemize}

%Gleichungen

\begin{equation}
\Delta p = p_\text{c} - p_\text{a} = 3.987 \ \text{bar} \\
\label{eqn:deltap}
\end{equation}


% Hier noch ein Tabellen Beispiel. Für Tabellen empfehle ich aber https://www.tablesgenerator.com/

\begin{table}[H]
\centering
\resizebox{\textwidth}{!}{%
\begin{tabular}{|c|c|c|c|c|}
\hline
\textbf{Cylinder} & $p_\text{c,max}$ / bar & $p_\text{Derform}$ / bar & $p_\text{Fail}$ / bar & Impermeability / true,false \\ \hline
\multicolumn{5}{|c|}{\textit{All at Once}}     \\ \hline
\textbf{Cylinder 1}       & 5  &     & 0   & $\times$ \\ \hline
\textbf{Cylinder 2}       & 5  &     & 0   & $\times$ \\ \hline
\textbf{Cylinder 3}       & 5  &     & 1   & $\times$ \\ \hline
\textbf{Cylinder 4}       & 5  &     & 1   & $\times$ \\ \hline
\multicolumn{5}{|c|}{\textit{Once at a Time}}  \\ \hline
\textbf{Cylinder 5}       & 5  &     & 1   & $\times$ \\ \hline
\textbf{Cylinder 6}       & 5  &     & 1   & $\times$ \\ \hline
\textbf{Cylinder 7}       & 5  &     & 1   & $\times$ \\ \hline
\textbf{Cylinder 8}       & 5  &     & 1.1 & $\times$ \\ \hline
\multicolumn{5}{|c|}{\textit{Optimized Printer Settings}} \\ \hline
\textbf{Cylinder 9}       & 5  &     & 1.2 & $\times$ \\ \hline
\textbf{Cylinder 10}      & 5  &     & 1.3 & $\times$ \\ \hline
\textbf{Cylinder 11}      & 5  &     & 1.5 & $\times$ \\ \hline
\multicolumn{5}{|c|}{\textit{Post-Processed}} \\ \hline
\textbf{Cylinder 8-PP}    & 6  & 3   & 4   & $\checkmark$ \\ \hline
\textbf{Cylinder 10-PP}   & 6  & 3.3 & 4   & $\checkmark$ \\ \hline
\textbf{Cylinder 11-PP}   & 6  & 5.2 & -   & $\checkmark$ \\ \hline
\end{tabular}%
}
\caption{xxxx}
\label{tab:airimpermeability}
\end{table}
