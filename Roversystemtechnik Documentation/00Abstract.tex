\chapter*{Danksagung}
Zunächst möchte ich mich im besonderen Maße bei meinen beiden Betreuern Felix und Manfred bedanken, welche mir jeder Zeit mit Rat und Tat zur Seite standen, egal um welche Uhrzeit. Vielen Dank für die Betreuung und Begleitung über diese Monate. Und natürlich bedanke ich mich auch bei Dr.-Ing.  Georg Herdrich, welcher mir als Prüfer dieser Arbeit erst ermöglichte, diese zu absolvieren.\\

Außerdem möchte ich mich noch bei Saskia, Maximilian und Nicolas bedanken, welche gleichzeitig mit mir ihre Bachelorarbeiten im Zuge des FerrAC Projekts absolviert haben und mir immer wieder unter die Arme gegriffen haben.\\

Für ihre Unterstützung und ihren Beistand während meines gesamten Studiums möchte ich mich auch nochmal herzlich bei meiner Familie bedanken. Und vor allem bei Stefanie und Leonard für das Korrekturlesen dieser Arbeit. \\

Zu guter Letzt möchte ich mich noch bei meiner Freundin Kristine bedanken, die mich in meinem Studium immer unterstützt, für mich da ist und jede noch so lange Nachtschicht mit mir mitmacht.\\

\newpage

\chapter*{Kurzfassung}
In dieser Bachelorarbeit wird die Entwicklung und der Aufbau eines Teststands zur Untersuchung von Drehmomenten beschrieben. Dabei wird diese als Teil des FerrAC (\textit{Ferrofluid Attitude Control}) Projekts am IRS (\textit{Inistut for Space Systems}) an der Universität Stuttgart durchgeführt. FerrAC hat als Ziel die Entwicklung eines neuartigen Aktuatortypes, welcher mit Hilfe von Ferrofluid Manipulation in der Lage sein soll ein Lageregelungssystem eines Satelliten, ohne jegliche mechanische bewegliche Teile zu ermöglichen. Innerhalb des Aktuators wird Ferrofluid durch zeitlich veränderbare magnetische Felder in eine Rotationsbewegung versetzt, welche in einem entstehenden Drehmoment und Drehimpuls resultiert. Diese können auf dem Satelliten dazu verwendet werden, um diesen auszurichten oder zu stabilisieren. Für die Validierung dieser Funktionen wird ein Teststand benötigt, welcher dazu in der Lage ist die Drehmomenterzegung des Aktuators zuverlässlich und quantifizierbar zu messen.\\

Diese Arbeit befasst sich daher mit einem Teststandentwurf, welches diese Aufgabe ermöglicht. Dabei ist die Simulation der Mikrogravitationsumgebung des Weltalls ausschlaggebend für die Entwicklung. Da die Aktuatoren für den Betrieb im Weltall ausgelegt sind, ist die Erdgravitation ein Störfaktor, welcher zu einer Fehlerquelle durch die daraus resultierende Reibung für alle Validierungen auf der Erde führt. Daher muss der Aspekt der Mikrogravitationsumgebung, welcher im Weltall die reibungsfreie Operation von Aktuatoren ermöglicht, für Teststände auf der Erde nachgestellt werden. Um diesen Effekt auszugleichen wurde ein Luftlager, welche auf Grund seiner Eigenschaften eine nahezu reibungsfreie Lagerung ermöglicht, entwickelt. Auf diesem Luftlager baut die Testplattform auf, welche die Montage diverser Aktuatoren zur Validierung ermöglicht. Für die Messung des Drehmoments wurde ein hoch präziser inkrementeller magnetischer Encoder ausgewählt, welcher in der Lage ist, die Winkelposition der Plattform möglichst exakt zu bestimmen. Aus dieser Winkelposition lassen sich mit Hilfe eines geeigneten Mikrocontrollers das Drehmoment und der Drehimpuls des Systems berechnen. Für die Validierung und die Simulation diverser Testszenarien ist des Weiteren ein Kontrolldrehmoment Mechanismus eingebaut, welcher als Aktuator ein herkömmliches Reaktionsrad verwendet und in der Lage ist ein vorgegebenes Drehmoment auf das System wirken zu lassen. Alle relevanten Design Entscheidungen, so wie Komponentenvergleiche werden im anschließenden dokumentiert, begründet und erklärt.\\

Die durchgeführten Experimente und Validierungen demonstrieren, dass die Entwicklung des Teststandes erfolgreich war und dieser die präzise Validierung von drehmomentgenerierenden Aktuatoren ermöglicht. Die nahezu Reibungsfreiheit, so wie die Drehmomentmessung wurden mit genügender Genauigkeit belegt. Dabei besitzt der Teststand eine Tragkraft von W$= 10$ kg bei einem Betriebsdruck im Bereich von $p_\text{o} = 1.25 - 1.5$ bar. Die Oberseite des Teststandes bietet eine Fläche mit einem Durchmesser von $D_\text{Pl} = 400$ mm für die Montage von Aktuatoren. Darüber hinaus, sind Drehmomentmessungen mit einer Genauigkeit von $T_\text{meas} = 0.00093$ Nm möglich. Damit demonstriert diese Arbeit die Durchführbarkeit der Entwicklung und Herstellung eines Luftlager Teststandes, welche in der Regel äußerst komplex und kostspielig sein kann, in einem universitärem Rahmen mit sowohl zeitlich als auch finanziell limitierten Ressourcen.


\newpage

\chapter*{Abstract}
In this bachelor thesis the design and manufacture of a test-bed for the investigation of torque generation is described. This is conducted as part of the FerrAC (\textit{Ferrofluid Attitude Control}) project at the IRS (\textit{Institute for Space Systems}) of the University of Stuttgart. FerrAC aims on developing a new type of actuator, which will be able to provide a satellite attitude control system without any mechanical moving parts by means of ferrofluid manipulation. Within the actuator, ferrofluid is rotated by time-varying magnetic fields, resulting in a torque and angular momentum that can be used to align or stabilize the satellite. The validation of these functionalities requires a test-bed capable of reliably and quantifiably measuring the torque generation of the actuator.\\

Therefore, this thesis deals with a design which makes this challenge possible. The simulation of the microgravity environment of space is crucial for the development. Since these actuators are engineered for operation in space, gravity on Earth leads to a source of error due to the resulting friction for all validations on Earth. Therefore, the aspect of the microgravity environment, which enables the frictionless operation of actuators in space, has to be simulated for test-beds on earth. To compensate this effect, an air bearing was developed, which due to its properties allows an almost friction-free bearing. The test platform is based on this air bearing, which enables the mounting of various actuators for validation. For the torque measurement, a high-precision incremental magnetic encoder was selected, which is able to determine a highly accurate angular position of the platform. From this angular position the torque and angular momentum of the system can be calculated with the help of a suitable microcontroller. For the validation and simulation of various test scenarios, a control torque mechanism is integrated, which uses a conventional reaction wheel as an actuator and is able to apply a defined torque to the system. All relevant design decisions, such as component comparisons, are documented, justified and explained in detail.\\

The experiments and validations performed demonstrate that the development of the test-bed  was a success as it enables the precise validation of torque-generating actuators. The almost frictionless system, as well as the torque measurement, were verified with sufficient accuracy. The test-bed possesses a load capacity of W$= 10$ kg at an operating pressure in the range of $p_\text{o} = 1.25 - 1.5$ bar. The top of the test-bed offers a surface with a diameter of $D_\text{Pl} = 400$ mm for the mounting of actuators. In addition, torque measurements with an accuracy of $T_\text{meas} = 0.00093$ Nm are possible. Thus, this thesis demonstrates the feasibility of developing and manufacturing an air bearing test-bed, which typically involves complex and costly work, in a university setting with limited resources, both in terms of time and money.
