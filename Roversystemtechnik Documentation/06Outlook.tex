\chapter{Outlook \& Risk Assessment}
\label{chap:outlook}

\subsection{Risk Assessment}

To reduce risks and increase the robustness within the development process of the INSPIRE mission, all subsystems focus on flight proven or space grade hard ware. \autoref{tab:SubSys-TRL} provides an overview on the TRL of each subsystem. 
 
\begin{table}[h]
\centering
\begin{tabular}{llll}
Subsystem            & overall TRL & Deviating component      &  \\ \hline\hline
Payload              & 0           & Drill System and Sample Analyser TRL 0 &  \\
Structure \& Mechan. & 9           & Boom Mechanism TRL 0     &  \\
Locomotion           & 9           & none                     &  \\
EPS                  & 9           & none                     &  \\
TCS                  & 9           & none                     &  \\
COMs, C\&DH          & 9           & Housekeeping TRL 0       & 	 \\ \hline
\end{tabular}
\caption{Subsystem TRL for risk assessment}
\label{tab:SubSys-TRL}
\end{table}

Housekeeping electronics and the Boom Mechanism to extend the camera head for the INSPIRE mission will be custom designed components and therefore rated at TRL 0. However due to proven development processes and simple testing the risk for the mission progress is rated non critical.  \\

Payload components development depict the highest risk for the mission. The Analyser for the ice core samples is a downsized replica of the [missing reference]. 
The ice core drill derives from the Nano Drill offered by Honeybee Robotics [Missing reference]. Although the technology is available, the drill has to be adapted and tested. Testing in analog conditions could take place in the arctic region and therefore lengthen the mission development process. \\

Thus the payload subsystem development contains the highest risk for the INSPIRE mission. However the risk is considered manageable. Additionally investing in payload development is regarded worth the risk as it contributes to the success of the Europa investigation.