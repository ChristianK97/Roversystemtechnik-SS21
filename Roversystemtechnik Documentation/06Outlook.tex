\chapter{Outlook \& Risk Assessment}
\label{chap:outlook}

\subsection{Risk Assessment}

To reduce risks and increase the robustnes within the development process of the INSPIRE mission, all subsystems focus on flight proven or space grade hard ware. \autoref{tab:SubSys-TRL} provides an overview on the TRL of each subsystem. 
 
\begin{table}[h]
\centering
\begin{tabular}{llll}
Subsystem            & overall TRL & Deviating component      &  \\ \hline\hline
Payload              & 0           & Drill and Analyser TRL 0 &  \\
Structure \& Mechan. & 9           & Boom Mechanism TRL 0     &  \\
Locomotion           & 9           & none                     &  \\
EPS                  & 9           & none                     &  \\
TCS                  & 9           & none                     &  \\
COMs, C\&DH          & 9           & Housekeeping TRL 0       & 	 \\ \hline
\end{tabular}
\caption{Subsystem TRL for risk assessment}
\label{tab:SubSys-TRL}
\end{table}

Housekeeping electronics for the INSPIRE mission will be newly developed and therefore rated at TRL 0. However due to proven development processes and simple testing the risk for the mission is rated non critical. 
In similar fashion the development of a Boom Mechanism to extend the camera head of the rover is rated non critical since it is based on proven technology. \\

Payload components development depict the highest risk for the mission. The Analyser for the ice core samples is a downsized replica of the [missing reference]. 
The ice core drill derives from the Nano Drill produced by Honeybee Robotics [Missing reference]. Although the technology is available, the drill has to be adapted and tested. Testing in analog conditions could take place in the arctic region and therefore lengthen the mission development process. \\

Thus the payload subsystem contains the highest risks for the INSPIRE mission. At the same time the mission benefits the most of an investment in payload development. 